W matematyce, grafy można zdefiniować jako graficzną reprezentację danych,
w której wartości są przedstawione w pewien uporządkowany sposób, zwykle w relacji do siebie nawzajem.
„\textit{Stanowią wygodny aparat do modelowania różnych obiektów, (...) i odpowiednio interpretowane
- mogą zawierać pewne informacje}”\cite{Wloch2008}.

Teoria grafów to dziedzina matematyki zajmująca się badaniem właściwości grafów,
będąca bardzo ważnym narzędziem w wielu „\textit{dziedzinach od rachunku operacyjnego, chemii, po genetykę, lingiwistykę
oraz od elektroniki i geografii po socjologię i architekturę}”\cite{Wilson2012}.
Grafy dają możliwość zobrazowania pewnych modeli, co jest szczególnie korzystne w analizie wzorców.
W kontekście grafów warto podkreślić ich zastosowanie poza teoretycznymi analizami.

W informatyce, grafy stanowią fundament wielu struktur danych i algorytmów, które umożliwiają efektywne przetwarzanie informacji.
Przykładem może być tutaj wyszukiwanie najkrótszej trasy w nawigacji GPS, gdzie węzły odpowiadają skrzyżowaniom, a krawędzie drogom.
W przypadku znajdowania najbardziej optymalnych tras, warto wymienić takie algorytmy jak A* czy Dijkstra.

W biologii, grafy pełnią ważną rolę w modelowaniu układu nerwowego oraz interakcji między genami.
W genetyce wykorzystuje się je do analizy drzew filogenetycznych, co pozwala chociażby na śledzenie ewolucji gatunków.

Natomiast w chemii, grafy służą do reprezentacji struktury molekularnej związków chemicznych,
umożliwiając naukowcom analizę ich właściwości i reaktywności.

W dziedzinie lingwistyki, grafy wykorzystywne są w modelowaniu struktury języka.
Wykorzystywane są, na przykład, w analizie morfologicznej czy syntaktycznej.
Dzięki nim możliwe jest również lepsze zrozumienie i przetwarzanie języka naturalnego przez komputery,
co stanowi podstawę technologii takich jak tłumaczenie automatyczne czy rozpoznawanie mowy.

W dziedzinie uczenia maszynowego i sztucznej inteligencji, grafy odgrywają kluczową rolę w reprezentacji danych i wiedzy.
Grafy wiedzy, które łączą różne informacje, umożliwiają tworzenie zaawansowanych systemów rekomendacyjnych i chatbotów.
Wykorzystanie grafów w analizie danych pozwala odkrywać ukryte zależności i wzorce, co jest niezwykle cenne w dzisiejszej erze big data

Rozpoznawanie wzorców, nam ludziom, pozwala na szybszą naukę przez rozpoznawanie czegoś, co już wcześniej widzieliśmy.
W bardzo dużym uproszczeniu, algorytmy uczenia maszynowego działają w podobny sposób.
Gdy model zostanie prawidłowo nauczony na pewnych danych, jest w stanie rozpoznawać podobne wzorce w innych,
nigdy wcześniej nie widzianych miejscach.
Grafy znajdują także zastosowanie w analizie sieci społecznych, gdzie pomagają w badaniu relacji między ludźmi,
takich jak przyjaźnie, współpraca zawodowa czy wpływy.
Dzięki analizie grafów można lepiej zrozumieć dynamikę grup społecznych i przepływ informacji,
co jest szczególnie istotne w marketingu i polityce.

Podsumowując, grafy są niezwykle wszechstronnym narzędziem, które znajduje zastosowanie w wielu dziedzinach nauki i technologii.
Ich zdolność do reprezentowania skomplikowanych struktur i relacji w sposób zrozumiały i przystępny jest nieoceniona.
Dzięki grafom możliwe jest analizowanie i przetwarzanie informacji w bardziej efektywny sposób. 

Celem pracy jest zobrazowanie owej zależności, na przykładzie nauczenia sieci neuronowej,
w taki sposób, by po wytrenowaniu na kilku typach grafów stworzonych sztucznie,
model był w stanie rozpoznać dane wzorce i je nazwać, w przestrzeni rzeczywistej.

\subsection{Przegląd literatury}
Lorem Ipsum is simply dummy text of the printing and typesetting industry. Lorem Ipsum has been the industry's standard dummy text ever since the 1500s, when an unknown printer took a galley of type and scrambled it to make a type specimen book. It has survived not only five centuries, but also the leap into electronic typesetting, remaining essentially unchanged. It was popularised in the 1960s with the release of Letraset sheets containing Lorem Ipsum passages, and more recently with desktop publishing software like Aldus PageMaker including versions of Lorem Ipsum.
