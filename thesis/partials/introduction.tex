Grafy w matematyce można zdefiniować jako graficzną reprezentację danych,
w której wartości są przedstawione w uporządkowany sposób,
zwykle w relacji do siebie nawzajem.
Teoria grafów to dziedzina matematyki zajmująca się badaniem właściwości grafów.
Jest to bardzo ważne narzędzie w wielu „\textit{dziedzinach od rachunku operacyjnego, chemii, po genetykę, lingiwistykę
oraz od elektroniki i geografii po socjologię i architekturę}”\cite{Wilson2012}.
Grafy dają możliwość zobrazowania pewnych modeli, co jest szczególnie korzystne w analizie wzorców.

Rozpoznawanie wzorców, nam ludziom, pozwala na szybszą naukę przez rozpoznawanie czegoś, co już wcześniej widzieliśmy.
W bardzo dużym uproszczeniu, algorytmy uczenia maszynowego działają w podobny sposób.
Gdy model zostanie prawidłowo nauczony na pewnych danych, jest w stanie rozpoznawać podobne wzorce w innych,
nigdy wcześniej nie widzianych miejscach.

Celem pracy jest zobrazowanie owej zależności, na przykładzie nauczenia sieci neuronowej,
w taki sposób, by po wytrenowaniu na kilku typach grafów stworzonych sztucznie,
model był w stanie rozpoznać dane wzorce i je nazwać, w przestrzeni rzeczywistej.

\subsection{Przegląd literatury}
Lorem Ipsum is simply dummy text of the printing and typesetting industry. Lorem Ipsum has been the industry's standard dummy text ever since the 1500s, when an unknown printer took a galley of type and scrambled it to make a type specimen book. It has survived not only five centuries, but also the leap into electronic typesetting, remaining essentially unchanged. It was popularised in the 1960s with the release of Letraset sheets containing Lorem Ipsum passages, and more recently with desktop publishing software like Aldus PageMaker including versions of Lorem Ipsum.
