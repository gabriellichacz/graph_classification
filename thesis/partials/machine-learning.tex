Uczenie maszynowe, znane również jako machine learning, to specjalistyczna gałąź sztucznej inteligencji,
która koncentruje się na konstruowaniu modeli i algorytmów umożliwiajcych komputerom samodzielne uczenie się z dostępnych danych.
W przeciwieństwie do systemów, które są bezpośrednio programowane do wykonania określonych zadań,
systemy uczenia maszynowego analizują dane, rozpoznają wzorce i podejmują decyzje oparte na zdobytej w ten sposób wiedzy.

\subsection{Rodzaje uczenia maszynowego}
Według \cite{Geron2020}, uczenie maszynowe można sklasyfikować na podstawie kilku kryteriów.
Jest to nadzór człowieka w procesie trenowania, możliwość modelu do uczenia się w czasie rzeczywistym
oraz sam sposób pracy (nauka z przykładów lub modelu). Kryteria te nie wykluczają się wzajemnie - można je dowolnie łączyć.
Za przykład może posłużyć filtr antyspamowy, który ciągle się uczy,
wykorzystując model sieci neuronowej i analizując wiadomości email.
Taki system można określić przyrostowym, opartym na modelu i nadzorowanym.

Dodatkowe kryteria oceny rodzaju uczenia maszynowego:
\begin{itemize}[label=-,labelsep=0.4cm,leftmargin=0.6cm]
    \item Uczenie nadzorowane (ang. supervised learning) to podejście, w którym model jest szkolony na danych,
        które są już odpowiednio oznaczone (np. rekordy mają przypisane odpowiednie klasy).
        Celem jest odkrycie funkcji, która przekształca dane wejściowe w oczekiwane wyjścia.
        Znajduje zastosowanie w klasyfikacji i regresji.
    \item Uczenie nienadzorowane (ang. unsupervised learning)
        - w tym przypadku model bada nieoznaczone dane, aby odkryć pewne wzorce lub struktury.
        Najczęściej stosowane w klasteryzacji, czy redukcji wymiarowości.
    \item Uczenie przez wzmacnianie (ang. reinforcement learning) to przypadek,
        gdzie model uczy się poprzez interakcję ze swoim otoczeniem,
        podejmując decyzje, które maksymalizują pewną nagrodę.
        Przykłady zastosowań to robotyka i gry komputerowe.
\end{itemize}