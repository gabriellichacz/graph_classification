\addcontentsline{toc}{section}{Literatura}

\begin{thebibliography}{4}
\bibitem{Wilson2012} Wilson R.J.: Wprowadzenie do teorii grafów. PWN, Warszawa 2012.
\bibitem{Wloch2008} Włoch A., Włoch I.: Matematyka dyskretna. Oficyna Wydawnicza Politechniki Rzeszowskiej, Rzeszów 2008.
\bibitem{Wojciechwoski2013} Wojciechowski J., Pieńkosz K.: Grafy i sieci. PWN, Warszawa 2013.
\bibitem{Fenner2020} Fenner M.E.: Uczenie maszynowe w Pythonie dla każdego. Helion SA, Gliwice 2020.
\bibitem{Geron2020} Géron A.: Uczenie maszynowe z użyciem Scikit-Learn i TensorFlow. Helion SA, Gliwice 2020.
\bibitem{Seenappa} Seenappa M.G.: Graph Classification using Machine Learning Algorithms. Master's Projects. 725, DOI: https://doi.org/10.31979/etd.b9pm-wpng, San Jose State University 2019.
\bibitem{str1} http://student.krk.pl/026-Ciosek-Grybow/rodzaje.html. Dostęp 26.03.2024.
\bibitem{str2} http://wms.mat.agh.edu.pl/\texttildelow md/ang-pol.pdf. Dostęp 29.03.2024. 
\bibitem{str3} https://cran.r-project.org/web/packages/igraph/index.html. Dostęp 10.03.2024.
\end{thebibliography}