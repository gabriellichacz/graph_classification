Model uczenia maszynowego jaki został wykorzystany w testach to sieć neuronowa.

Do testów stworzone zostało kilka modeli sieci neuronowych,
wytrenowanych na rysunkach grafów stworzonych za pomocą skryptów R.
Implementacja została wykonana biblioteką TensorFlow oraz Keras w języku Python.
Modele są w stanie rozpoznawać rysunki grafów i przypisywać im odpowiednie klasy.
Celem było również przetestowanie modeli na rzeczywistych zdjęciach,
zawierających wzorce przypominające grafy, bądź rysunkach grafów narysowanych ręcznie.

Klasy, których rozpoznawania uczony był model:
\begin{itemize}[label=-,labelsep=0.4cm,leftmargin=0.6cm]
	\item Graf bezkrawędziowy
	\item Graf pełny
	\item Drzewo binarne
	\item Ścieżka
	\item Cykl
\end{itemize}

Stworzone zostały 3 modele:
\begin{itemize}[label=-,labelsep=0.4cm,leftmargin=0.6cm]
	\item wytrenowany na danych ze stałą liczbą wierzchołków
	\item wytrenowany na danych ze stałą liczbą wierzchołków oraz walidacją krzyżową
	\item wytrenowany na danych ze zmienną liczbą wierzchołków
\end{itemize}

\subsection{Generacja danych}
Dane wygenerowane zostały przy pomocy skryptu stworzonego w języku R oraz biblioteki igraph.
Skrypt został zaprojektowany funkcyjnie, by osiągnąć możliwie największą automatyzację testów.
Rysunki grafów tworzone były o wielkości 800x600 pikseli, na białym tle, z wierzchołkami w kolorze pomarańczowym,
bez jakichkolwiek oznaczeń wierzchołków oraz zapisywane w odpowiednich katalogach, odpowiadających klasie grafu.
Przygotowane zostały funkcje tworzące ścieżki, cykle, grafy pełne, grafy bezkrawędziowe oraz drzewa binarne.
W każdej z funkcji możliwy jest wybór liczby generowanych grafów, liczba wierzchołków grafu
oraz współczynnik odpowiadający za zakrzywienie krawędzi na rysunkach.

\begin{lstlisting}[language=R,caption=Listing skryptu rysującego grafy,label={tests-generation-1}]
	#' Rysuj graf
	#'
	#' @param graph Graph - Graf do narysowania
	#' @param pathName string - Sciezka
	#' @param fileName string - Nazwa pliku
	#' @param vertexNo int - liczba wierzcholkow
	#' @param i int - Numer iteracji
	#' @param plotCurve float
	#' @return void
	#'
	plotGraphHelper <- function(graph, pathName, fileName, vertexNo, i, plotCurve)
	{
		path <- file.path(pathName, paste0(
			fileName, "-", vertexNo, "-", i, ".png"
		))
		png(path, width = 800, height = 600)
		plot(graph, vertex.label = NA, edge.curved = plotCurve)
		dev.off()
	}	
\end{lstlisting}

\begin{lstlisting}[language=R,caption=Listing funkcji tworzącej ścieżkę,label={tests-generation-2}]
	#' Graf sciezka N wierzcholkow, nieskierowany
	#'
	#' @param N int - liczba rysunkow
	#' @param vertexNo int - liczba wierzcholkow
	#' @return void
	#'
	plotPaths <- function(N, vertexNo)
	{
		fileName <- 'path'
		pathName <- createDir(vertexNo, fileName)
		definition <- c()
		for (index in 1:(vertexNo-1))
		{
			definition <- c(definition, index, index + 1)
		}
		definitionMatrix <- matrix(
			definition, ncol = 2, byrow = TRUE
		)
		
		for (i in 1:N)
		{
			plotCurve <- generateGaussian(0.01, 0.99)
			graph <- graph_from_edgelist(
				definitionMatrix, directed = FALSE
			)
			E(graph)$weight <- runif(ecount(graph))
			plotGraphHelper(
				graph, pathName, fileName, vertexNo, i, plotCurve
			)
		}
	}
\end{lstlisting}

W testach wykorzystane zostały wszystkie wybrane typy grafów.
Każdy z nich, czyli dana liczba wierzchołków i typ grafu, wygenerowany został w liczbie 500 sztuk.
Warianty liczby wierzchołków generowanych grafów to 4, 5, 6 oraz 7 wierzchołków.
Testy zostały przeprowadzone na dwa sposoby - ze stałą krzywizną krawędzi, wynoszącą 0,3,
oraz z losowym parametrem krzywizny krawędzi, mieszczącym się w przedziale od 0 do 1.
Testy ze stałą krzywizną wierzchołków nie zostały przedstawione w dalszej częście pracy,
ze względu na ich niską wartość merytoryczną. Wyniki uzyskane z tych testów nie były zadowalające.

\begin{figure}[ht]
	\centering
	\includegraphics[height=8cm]{resources/model/images/gen-graphs-generated.png}
	\caption{Przykładowe wygenerowane rysunki grafów z każdej klasy}
	\label{Fig:tests-generation-1}
\end{figure}
\FloatBarrier

\subsection{Dane zewnętrzne}
\begin{frame}
    \frametitle{Dane zewnętrzne}

    \begin{figure}[ht]
        \centering
        \includegraphics[height=3cm]{../thesis/resources/model/images/ext-graphs-drawn.png}
        \caption{Przykładowe zewnętrzne rysunki grafów narysowane odręcznie}
    \end{figure}
    
    \begin{figure}[ht]
        \centering
        \includegraphics[height=2cm]{../thesis/resources/model/images/ext-graphs-internet.png}
        \caption{Przykładowe zewnętrzne rysunki grafów pobrane z~internetu}
    \end{figure}

\end{frame}

\subsection{Opis skryptu}
\subsubsection{Przygotowanie}
Wszystkie przygotowane skrypty testowe rozpoczynają się od przygotowania środowiska do trenowania modelu.
Najpierw ustawiana jest ścieżka do katalogów z~wygenerowanymi grafami oraz do katalogów na~dane treningowe i~walidacyjne.
Następnie sprawdzane jest, czy te katalogi istnieją, a~jeśli nie, są tworzone.
Dalej, skrypty definiują parametry dotyczące wielkości obrazów oraz wielkości partii danych, które będą używane podczas treningu.
Dla każdej wartości liczby wierzchołków ustawiana jest ścieżka do katalogu z~wygenerowanymi grafami,
pobierana lista podkatalogów oraz obrazów w~każdym z~nich.
Następnie obrazy dzielone są na~zestawy treningowe i~walidacyjne w~stosunku 80:20.
Daje to~8 tys. grafów w~zbiorze uczącym i~2 tys. grafów w~zbiorze testowym.
W przypadku modeli wykorzystujących wszystkie warianty liczby wierzchołków, dane przenoszone są do jednego katalogu
i od razu dzielone na~zbiory treningowe i~walidacyjne.

\subsubsection{Model}
Każdy typ modelu tworzony jest w~inny sposób. Opisana zostanie tu główna zasada i~ich elementy wspólne.
Na początku, skrypt wczytuje obrazy przygotowane na~wcześniejszym etapie do odpowiednich zmiennych - treningowe i~walidacyjne.
W przypadku modeli z~walidacją krzyżową, dla każdej itreacji walidacyjnej, dane zostały podzielone inaczej.
Po wczytaniu danych, zostają one przeskalowane do wielkości 180x180 pikseli i~przekształcone do odcieni szarości.

\clearpage

\begin{lstlisting}[language=Python,caption=Listing skryptu tworzącego model z~walidacją krzyżową
	oraz uczonym na~wszystkich wariantach liczby wierzchołków grafów,label={tests-model-1}]
	n_splits = 5
	kfold = KFold(n_splits=n_splits, shuffle=True, random_state=42)
	history = []
	all_images = [os.path.join(dp, f) for dp, dn, filenames in os.walk(data_dir_model) for f in filenames if os.path.splitext(f)[1] == '.png']
  
	for train_index, val_index in kfold.split(all_images):
		train_images = [all_images[i] for i~in train_index]
		validation_images = [all_images[i] for i~in val_index]

		# Generowanie danych treningowych
		train_ds = tf.keras.preprocessing.image_dataset_from_directory(
		train_dir,
		image_size=(img_height, img_width),
		batch_size=batch_size)

		class_names = train_ds.class_names

		train_ds = train_ds.map(lambda x, y: (rgb_to_grayscale(x), y))

		# Generowanie danych walidacyjnych
		val_ds = tf.keras.preprocessing.image_dataset_from_directory(
		validation_dir,
		image_size=(img_height, img_width),
		batch_size=batch_size)

		val_ds = val_ds.map(lambda x, y: (rgb_to_grayscale(x), y))

		# Tworzenie modelu
		model = tf.keras.models.Sequential([
		tf.keras.layers.Rescaling(1./255),
		tf.keras.layers.Conv2D(32, 3, activation='relu'),
		tf.keras.layers.MaxPooling2D(),
		tf.keras.layers.Conv2D(32, 3, activation='relu'),
		tf.keras.layers.MaxPooling2D(),
		tf.keras.layers.Conv2D(32, 3, activation='relu'),
		tf.keras.layers.MaxPooling2D(),
		tf.keras.layers.Flatten(),
		tf.keras.layers.Dense(128, activation='relu', kernel_regularizer=tf.keras.regularizers.l2(0.01)),
		tf.keras.layers.Dropout(0.2),
		tf.keras.layers.Dense(len(class_names))
		])
		
		# Kompilacja modelu
		model.compile(
			optimizer='adam',
			loss=tf.losses.SparseCategoricalCrossentropy(from_logits=True),
			metrics=['accuracy']
		)

		# Uczenie modelu
		history.append(model.fit(
			train_ds,
			validation_data=val_ds,
			epochs=75
		))
\end{lstlisting}

Model sieci neuronowej został zdefiniowany jako sekwencyjny stos warstw.
Dla standaryzacji danego testu, w~przypadku modeli z~walidacją krzyżową, ustalono $k$-Fold z~liczbą podziałów równą 5.
Pierwsza warstwa to~warstwa Rescaling, która normalizuje wartości pikseli do zakresu [0, 1].
W przykładzie, parametr $1./255$ oznacza, że każda wartość piksela mnożona jest przez $\frac{1}{255}$.
Następne trzy warstwy to~Conv2D, z~których każda jest następowana warstwą MaxPooling2D.
W przykładzie, warstwa kolwolucyjna stosuje 32 filtry o~wymiarach 3x3 oraz funkcję aktywacji ReLU,
która wprowadza nieliniowość do modelu.
MaxPooling2D redukuje rozmiar danych wejściowych,
wybierając maksymalną wartość z~każdego regionu (domyślnie oraz tutaj - 2x2). 
Po wyżej wymienionych warstwach, znajduje się warstwa Flatten, która przekształca mapy cech 2D w~wektor 1D.
Innymi słowy, przekształca wielowymiarową macierz wyjściową z~poprzedniej warstwy do jednowymiarowego wektora.
Następnie, dodana jest w~pełni połączona (Dense) warstwa z~128 neuronami
i funkcją aktywacji, podobnie jak w~przypadku Conv2D, ReLU.
Wprowadzona jest również regularyzacja L2, która dodaje karę za duże wartości wag,
by zmniejszyć ryzyko przeuczenia. Została zastosowana z~siłą 0,01.
Kolejna warstwa to~Droput, która losowo wyłącza 20\% neuronów podczas uczenia,
co również jest moetodą zapobiegającą przeuczeniu.
Warstwa wyjściowa zawiera tyle jednostek, ile występuje klas w~danych uczących.
Zależnie od danego testu, może być to~różna liczba.
W przypadku warstw konwolucyjnych, wybrano 32 filtry, a~dla warstwy w~pełni połączonej zastosowano 128 jednostek.
Liczba epok w~podstawowej wersji modelu wyniosła 75.

W kolejnych wariantach modeli, zmieniane były parametry poszczególnych warstw, funkcje aktywacji,
czy również same warstwy, w~celu znalezienia najbardziej optymalnej kombinacji.

\subsubsection{Wyniki}
Po wytrenowaniu modelu, skrypt dokonuje wizualizacji dokładności i~straty modelu.
Najpierw wyświetla w~konsoli wartości dokładności dla obu zbiorów z~historii treningu.
Dalej tworzy wykresy, gdzie na~pierwszym z~nich pokazuje dokładność na~zbiorze treningowym i~walidacyjnym,
a na~drugim wykresie prezentuje stratę modelu dla obu zbiorów.
Jednostką straty jest entropia krzyżowa (cross-entropy), która jest wyrażana jako liczba bezwzględna.
Entropia krzyżowa mierzy różnicę między rzeczywistymi etykietami a~przewidywanymi prawdopodobieństwami klas.
Im mniejsza wartość entropii krzyżowej, tym lepiej model przewiduje klasy.
Dokładność jest wyrażana jako wartość procentowa lub~ułamek, gdzie 1 oznacza 100\% dokładności.
Na przykład, jeśli model przewiduje poprawnie 90 na~100 przypadków, dokładność wynosi 0.9 lub~90\%.

\begin{figure}[ht]
	\centering
	\includegraphics[width=15.5cm]{resources/model/images/scr-standard-result.png}
	\caption{Przykładowe wartości dokładności dla zbioru treningowe i~walidacyjnego}
	\label{Fig:tests-wyniki-2}
\end{figure}
\FloatBarrier

\begin{figure}[ht]
	\centering
	\includegraphics[height=6cm]{resources/model/images/v2_epoch75.png}
	\caption{Przykładowa wizualizacja dokładności i~straty wytrenowanego modelu}
	\label{Fig:tests-wyniki-1}
\end{figure}
\FloatBarrier

\subsubsection{Testy na~danych zewnętrznych}
Po wyświetleniu dokładnosci modelu skrypt przeszukuje katalog z~danymi i~jego podkatalogi, by przygotować obrazy zewnętrzne.
Następnie ustawia ścieżkę do katalogu z~obrazami testowymi i~pobiera ich listę.
Dla każdego obrazu w~tej liście wczytuje go, przeskalowuje do odpowiedniego rozmiaru i~konwertuje do skali szarości
Następnie model przewiduje klasę obrazu, a~wynik jest wyświetlany w~konsoli.

\subsection{Testy modeli}
\subsubsection{Model podstawowy}

\begin{figure}[ht]
	\centering
	\includegraphics[height=5.5cm]{partials/images/tests/v2_epoch75.png}
	\caption{Wyniki testów dla modelu podstawowego}
	\label{Fig:tests-base-1}
\end{figure}
\FloatBarrier

\begin{figure}[ht]
	\centering
	\includegraphics[height=7cm]{partials/images/tests/v2_epoch75_img_tests.png}
	\caption{Klasyfikacja obrazów zewnętrznych dla modelu ze zmienną liczbą wierzchołków}
	\label{Fig:tests-base-2}
\end{figure}
\FloatBarrier

\subsubsection{Model z walidacją krzyżową}

W przypadku standardowego modelu z walidacją krzyżową model bardzo szybko uległ przeuczeniu.
Już po szóstej iteracji dokładność na zbiorze walidaycjnym wyniosła 100\%, co nie jest realistycznie możliwe.
Została podjęta próba ograniczenia przeuczenia poprzez zwiększenie zbioru danych, zmiany liczby epok w modelu
oraz manipulacji współczynnikami dropout i regularyzacji.
W każdym przypadku model zwracał niezadowalające wyniki wynoszące 100\% po jednej z początkowych iteracji.

\begin{figure}[ht]
	\centering
	\includegraphics[height=5.5cm]{partials/images/tests/v2_crossvalid.png}
	\caption{Wyniki testów dla modelu z walidacją krzyżową}
	\label{Fig:tests-cv-1}
\end{figure}
\FloatBarrier

Z powodu przeuczenia model nie radził sobie z zewnętrznymi obrazkami testowymi.
Większość grafów określił jako grafy pełne, co nie jest zgodne ze stanem rzeczywistym.

\begin{figure}[ht]
	\centering
	\includegraphics[height=7cm]{partials/images/tests/v2_crossvalid_img_tests.png}
	\caption{Klasyfikacja obrazów zewnętrznych dla modelu z walidacją krzyżową}
	\label{Fig:tests-cv-2}
\end{figure}
\FloatBarrier

\subsubsection{Model ze zmienną liczbą wierzchołków}
Najlepsze wyniki pod względem rozpoznawania zewnętrznych obrazków testowych
oraz realistycznej dokładności na zbiorze walidacyjnym,
zostały uzyskane przy użyciu modelu sieci neuronowej uczonej na rysunkach grafów z różną liczbą wierzchołków.
Było to odpowiednio 4, 5, 6 oraz 7 wierzchołków.

\begin{figure}[ht]
	\centering
	\includegraphics[height=5.5cm]{partials/images/tests/v2_multiple_edges_epoch75.png}
	\caption{Wyniki testów dla modelu ze zmienną liczbą wierzchołków}
	\label{Fig:tests-var-1}
\end{figure}
\FloatBarrier

Model nie poradził sobie zbyt dobrze z obrazami zewnętrznymi, lecz znacznie lepiej niż model z walidacją krzyżową.
Poprawnie wskazanych klas grafów było 5 z 14 wszystkich rysunków.
Mimo, że model jest w stanie poprawnie określić niektóre typy grafów poprawnie,
wciąż jest to dokładność niższa niż 50\%.

\begin{figure}[ht]
	\centering
	\includegraphics[height=7cm]{partials/images/tests/v2_multiple_edges_epoch75_img_tests.png}
	\caption{Klasyfikacja obrazów zewnętrznych dla modelu z walidacją krzyżową}
	\label{Fig:tests-var-2}
\end{figure}
\FloatBarrier

\subsubsection{Model ze zmienną liczbą wierzchołków i walidacją krzyżową}

\begin{figure}[ht]
	\centering
	% \includegraphics[height=5.5cm]{partials/images/tests/v2_multiple_edges_crossvalid.png}
	\caption{Wyniki testów dla modelu ze zmienną liczbą wierzchołków i walidacją krzyżową}
	\label{Fig:tests-csvar-1}
\end{figure}
\FloatBarrier

\begin{figure}[ht]
	\centering
	% \includegraphics[height=5.5cm]{partials/images/tests/v2_multiple_edges_crossvalid_img_tests.png}
	\caption{Klasyfikacja obrazów zewnętrznych dla modelu ze zmienną liczbą wierzchołków i walidacją krzyżową}
	\label{Fig:tests-csvar-2}
\end{figure}
\FloatBarrier

\subsection{Wnioski}
W przypadku uczenia modeli z wykorzystaniem grafów pełnych, najczęściej dominowały one cały zbiór danych,
przez co modele w kolejnych testach klasyfikowały większość testowych grafów rysowanych odręcznie jako właśnie grafy pełne.

Testy z wykorzystaniem stałej liczby wierzchołków grafów okazały się mniej owocne niż testy z rysunkami grafów
o zmiennej liczbie wierzchołków.

Wystąpiła tendencja do niepoprawnego określania innych grafów, grafami dwudzielnymi, jeśli takie znajdowały się
w zbiorze danych treningowych.