Model uczenia maszynowego jaki został wykorzystany w testach to sieć neuronowa 

Do testów stworzone zostało kilka modeli sieci neuronowych,
wytrenowanych na rysunkach grafów stworzonych za pomocą skryptów R.
Implementacja została wykonana biblioteką TensorFlow oraz Keras w języku Python.
Modele są w stanie rozpoznawać rysunki grafów i przypisywać im odpowiednie klasy.
Celem było również przetestowanie modeli na rzeczywistych zdjęciach,
zawierających wzorce przypominające grafy, bądź rysunkach grafów narysowanych ręcznie.

Stworzone zostały 3 modele:
- wytrenowany na danych ze stałą liczbą wierchołków
- wytrenowany na danych ze stałą liczbą wierchołków oraz walidacją krzyżową
- wytrenowany na danych ze zmienną liczbą wierzchołków

\subsection{Wyniki}
Lorem Ipsum is simply dummy text of the printing and typesetting industry. Lorem Ipsum has been the industry's standard dummy text ever since the 1500s, when an unknown printer took a galley of type and scrambled it to make a type specimen book. It has survived not only five centuries, but also the leap into electronic typesetting, remaining essentially unchanged. It was popularised in the 1960s with the release of Letraset sheets containing Lorem Ipsum passages, and more recently with desktop publishing software like Aldus PageMaker including versions of Lorem Ipsum.

\subsection{Wnioski}
Lorem Ipsum is simply dummy text of the printing and typesetting industry. Lorem Ipsum has been the industry's standard dummy text ever since the 1500s, when an unknown printer took a galley of type and scrambled it to make a type specimen book. It has survived not only five centuries, but also the leap into electronic typesetting, remaining essentially unchanged. It was popularised in the 1960s with the release of Letraset sheets containing Lorem Ipsum passages, and more recently with desktop publishing software like Aldus PageMaker including versions of Lorem Ipsum.
