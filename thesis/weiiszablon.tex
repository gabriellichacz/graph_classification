\documentclass[12pt,twoside]{article}

\usepackage{weiiszablon}
\usepackage{placeins}
\usepackage{tikz} % Do używania zmiennych przy numerowaniu definicji - można usunąć na samym końcu i ręcznie ponumerować
\usepackage{listings}
\usepackage{amssymb}

\author{Gabriel Lichacz}
\studentID{164174}

\title{Rozpoznawanie rysunków grafów}
\titleEN{Recognition of graphs}

\newcommand{\rodzajPracyNo}{2}

%%% promotor
\supervisor{dr Paweł Bednarz}
\supervisorEN{Paweł Bednarz PhD}

\abstract{Treść streszczenia po polsku}
\abstractEN{Treść streszczenia po angielsku}

\begin{document}

% strona tytułowa
\maketitle

\blankpage

% spis treści
\tableofcontents

\clearpage
\blankpage

\section*{Wykaz symboli}
$G$ - graf

$V(G)$ - zbiór wierzchołków grafu $G$

$E(G)$ - zbiór krawędzi grafu $G$

$D$ - digraf

$K_n$ - graf pełny

$N_n$ - graf bezkrawędziowy

$C_n$ - cykl $n$-wierzchołkowy

$P_n$ - ścieżka $n$-wierzchołkowa

$deg(v)$ - stopień wierzchołka $v$

$T$ - drzewo

$v_0 \rightarrow ... \rightarrow v_m$ - droga w grafie

% -- TO DO -- % More
\clearpage

\section{Wstęp}
\begin{frame}
    \frametitle{Wstęp}
    This is some text in the first frame. This is some text in the first frame. This is some text in the first frame.
\end{frame}

\section{Podstawowe definicje teorii grafów}
\subsection{Definicje}
Definicje zostały zaczerpnięte z literatury, z pozycji \cite{Wilson2012}, \cite{Wloch2008} oraz \cite{Wojciechwoski2013}.

\newcommand{\graphDefinitionIndex}{1}
\newcommand{\incrementGraphDefinitionIndex} {
    \pgfmathtruncatemacro{\graphDefinitionIndex}{\graphDefinitionIndex + 1}
}

\noindent
\textbf{Definicja \graphDefinitionIndex.}
\incrementGraphDefinitionIndex
Grafem nieskierowanym, skończonym G nazywamy parę $(V,E)$, gdzie $V = V(G)$ jest zbiorem skończonym, niepustym,
natomiast $E = E(V)$ jest rodziną mogących się powtarzać dwuelementowych podzbiorów niekoniecznie różnych elementów ze zbioru $V$.
Zbiór $V(G)$ nazywamy zbiorem wierzchołków (lub węzłami), a elementy tego zbioru nazywamy wierzchołkami i oznaczamy symbolami:
$x$, $y$, $x_i$, $y_i$, $1$, $2$, ... Zbiór $E(G)$ nazywamy zbiorem krawędzi grafu $G$.
Mówimy, że krawędź $\{v, w\}$ łączy wierzchołki $v$ i $w$, i na ogół oznaczamy ją krócej symbolem $vw$.

\noindent
\textbf{Definicja \graphDefinitionIndex.}
\incrementGraphDefinitionIndex
W wielu zagadnieniach nazwy wierzchołków są nieistotne, więc je pomijamy i mówimy wtedy, że graf jest nieoznakowany.

\noindent
\textbf{Definicja \graphDefinitionIndex.}
\incrementGraphDefinitionIndex
Jeżeli w grafie G istnieją co najmniej dwie krawędzie $\{x, y\}$, to krawędź tę nazywamy krawędzią wielokrotną.

\noindent
\textbf{Definicja \graphDefinitionIndex.}
\incrementGraphDefinitionIndex
Krawędź $\{x, x\}$ w grafie G nazywamy pętlą.

\noindent
\textbf{Definicja \graphDefinitionIndex.}
\incrementGraphDefinitionIndex
Graf mający krawędzie wielokrotne nazywamy multigrafem.

\noindent
\textbf{Definicja \graphDefinitionIndex.}
\incrementGraphDefinitionIndex
Graf, który nie ma krawędzi wielokrotnych i pętli, nazywamy grafem prostym.

\noindent
\textbf{Definicja \graphDefinitionIndex.}
\incrementGraphDefinitionIndex
Graf zawierający pętle nazywamy pseudografem.

\noindent
\textbf{Definicja \graphDefinitionIndex.}
\incrementGraphDefinitionIndex
Graf $G$ taki, że $E(G) = \emptyset$, nazywamy grafem pustym. Jeżeli $|V(G)| = n$, to graf pusty oznaczony symbolem $N_n$.
Każdy wierchołek grafu pustego jest wierchołkiem izolowanym.

\noindent
\textbf{Definicja \graphDefinitionIndex.}
\incrementGraphDefinitionIndex
Graf prosty $G$ taki, że każde dwa wierzchołki są sąsiednie, nazywamy grafem pełnym.
Jeżeli $|V(G)| = n$, to graf pełny oznaczamy $K_n$.

\noindent
\textbf{Definicja \graphDefinitionIndex.}
\incrementGraphDefinitionIndex
Graf G, którego zbiór wierchołków można podzielić na dwa rozłączne, niepuste podzbiory $V_1$ i $V_2$ tak,
że jeżeli $\{x, y\} \in E(G)$, to $x \in V_1 \vee y \in V_2$ nazywamy grafem dwudzielnym.

\begin{figure}[ht]
	\centering
	\includegraphics[height=11cm]{partials/images/graph_directed.png}
	\caption{Przykład grafu skierowanego}
	\label{Fig:graphs-undirected-1}
\end{figure}
\FloatBarrier

\begin{figure}[ht]
	\centering
	\includegraphics[height=11cm]{partials/images/graph_undirected.png}
	\caption{Przykład grafu nieskierowanego}
	\label{Fig:graphs-directed-1}
\end{figure}
\FloatBarrier

\section{Uczenie maszynowe}
\begin{frame}
    \frametitle{Uczenie maszynowe}
    This is some text in the first frame. This is some text in the first frame. This is some text in the first frame.
\end{frame}

\section{Wykorzystywane technologie}
\begin{frame}
    \frametitle{Wykorzystywane technologie}
    This is some text in the first frame. This is some text in the first frame. This is some text in the first frame.
\end{frame}

\section{Testy}
\subsubsection{Model podstawowy}

\begin{figure}[ht]
	\centering
	\includegraphics[height=5.5cm]{partials/images/tests/v2_epoch75.png}
	\caption{Wyniki testów dla modelu podstawowego}
	\label{Fig:tests-base-1}
\end{figure}
\FloatBarrier

\begin{figure}[ht]
	\centering
	\includegraphics[height=7cm]{partials/images/tests/v2_epoch75_img_tests.png}
	\caption{Klasyfikacja obrazów zewnętrznych dla modelu ze zmienną liczbą wierzchołków}
	\label{Fig:tests-base-2}
\end{figure}
\FloatBarrier

\subsubsection{Model z walidacją krzyżową}

W przypadku standardowego modelu z walidacją krzyżową model bardzo szybko uległ przeuczeniu.
Już po szóstej iteracji dokładność na zbiorze walidaycjnym wyniosła 100\%, co nie jest realistycznie możliwe.
Została podjęta próba ograniczenia przeuczenia poprzez zwiększenie zbioru danych, zmiany liczby epok w modelu
oraz manipulacji współczynnikami dropout i regularyzacji.
W każdym przypadku model zwracał niezadowalające wyniki wynoszące 100\% po jednej z początkowych iteracji.

\begin{figure}[ht]
	\centering
	\includegraphics[height=5.5cm]{partials/images/tests/v2_crossvalid.png}
	\caption{Wyniki testów dla modelu z walidacją krzyżową}
	\label{Fig:tests-cv-1}
\end{figure}
\FloatBarrier

Z powodu przeuczenia model nie radził sobie z zewnętrznymi obrazkami testowymi.
Większość grafów określił jako grafy pełne, co nie jest zgodne ze stanem rzeczywistym.

\begin{figure}[ht]
	\centering
	\includegraphics[height=7cm]{partials/images/tests/v2_crossvalid_img_tests.png}
	\caption{Klasyfikacja obrazów zewnętrznych dla modelu z walidacją krzyżową}
	\label{Fig:tests-cv-2}
\end{figure}
\FloatBarrier

\subsubsection{Model ze zmienną liczbą wierzchołków}
Najlepsze wyniki pod względem rozpoznawania zewnętrznych obrazków testowych
oraz realistycznej dokładności na zbiorze walidacyjnym,
zostały uzyskane przy użyciu modelu sieci neuronowej uczonej na rysunkach grafów z różną liczbą wierzchołków.
Było to odpowiednio 4, 5, 6 oraz 7 wierzchołków.

\begin{figure}[ht]
	\centering
	\includegraphics[height=5.5cm]{partials/images/tests/v2_multiple_edges_epoch75.png}
	\caption{Wyniki testów dla modelu ze zmienną liczbą wierzchołków}
	\label{Fig:tests-var-1}
\end{figure}
\FloatBarrier

Model nie poradził sobie zbyt dobrze z obrazami zewnętrznymi, lecz znacznie lepiej niż model z walidacją krzyżową.
Poprawnie wskazanych klas grafów było 5 z 14 wszystkich rysunków.
Mimo, że model jest w stanie poprawnie określić niektóre typy grafów poprawnie,
wciąż jest to dokładność niższa niż 50\%.

\begin{figure}[ht]
	\centering
	\includegraphics[height=7cm]{partials/images/tests/v2_multiple_edges_epoch75_img_tests.png}
	\caption{Klasyfikacja obrazów zewnętrznych dla modelu z walidacją krzyżową}
	\label{Fig:tests-var-2}
\end{figure}
\FloatBarrier

\subsubsection{Model ze zmienną liczbą wierzchołków i walidacją krzyżową}

\begin{figure}[ht]
	\centering
	% \includegraphics[height=5.5cm]{partials/images/tests/v2_multiple_edges_crossvalid.png}
	\caption{Wyniki testów dla modelu ze zmienną liczbą wierzchołków i walidacją krzyżową}
	\label{Fig:tests-csvar-1}
\end{figure}
\FloatBarrier

\begin{figure}[ht]
	\centering
	% \includegraphics[height=5.5cm]{partials/images/tests/v2_multiple_edges_crossvalid_img_tests.png}
	\caption{Klasyfikacja obrazów zewnętrznych dla modelu ze zmienną liczbą wierzchołków i walidacją krzyżową}
	\label{Fig:tests-csvar-2}
\end{figure}
\FloatBarrier

\section{Podsumowanie i wnioski końcowe}
\begin{frame}
    \frametitle{Podsumowanie}

    \begin{figure}[ht]
        \centering
        \includegraphics[width=10cm]{../thesis/resources/tests/images/v3/base6_img.png}
        \caption{Wyniki testów dla modelu podstawowego, liczba wierzchołków n = 6}
    \end{figure}

    \begin{figure}[ht]
        \centering
        \includegraphics[width=10cm]{../thesis/resources/tests/images/v3/base6_bar.png}
        \caption{Wizualizacja klasyfikacji obrazów zewnętrznych dla modelu podstawowego, liczba wierzchołków n = 6}
    \end{figure}

\end{frame}

\clearpage

\section*{Załączniki}
\addcontentsline{toc}{section}{Załączniki}

\begin{itemize}[label=-,labelsep=0.4cm,leftmargin=0.6cm]
    \item Skrypt generujący obrazy grafów
    \item Skrypty testowe z modelem podstawowym
    \item Skrypty testowe z modelem, z walidacją krzyżową
    \item Skrypty testowe z modelem dostosowanym do nauki grafów o różnej liczbie wierzchołków
    \item Skrypty testowe z modelem, z walidacją krzyżową, dostosowanym do nauki grafów o różnej liczbie wierzchołków
    \item Zbiór zewnętrznych grafów testowych
\end{itemize}

\vspace*{\fill}

Wersja online dokumentu, dane oraz kod źródłowy w języku Python i R dostępne na stronie:
https://github.com/gabriellichacz/graph\_classification

\clearpage

\addcontentsline{toc}{section}{Literatura}

\begin{thebibliography}{4}
\bibitem{Wilson2000} Wilson R.J.: Wprowadzenie do teorii grafów. PWN, Warszawa 2000.
\bibitem{str} http://student.krk.pl/026-Ciosek-Grybow/rodzaje.html. Dostęp 26.03.2024.
\bibitem{str} http://wms.mat.agh.edu.pl/\texttildelow md/ang-pol.pdf. Dostęp 29.03.2024. 
\bibitem{str} https://cran.r-project.org/web/packages/igraph/index.html. Dostęp 10.03.2024.
\end{thebibliography}

\clearpage

\makesummary

\end{document}