\documentclass[12pt,twoside]{article}

\usepackage{weiiszablon}

\author{Gabriel Lichacz}
\studentID{164174}

\title{Rozpoznawanie rysunków grafów}
\titleEN{Recognition of graphs}

\newcommand{\rodzajPracyNo}{2}

%%% promotor
\supervisor{dr Paweł Bednarz}
\supervisorEN{Paweł Bednarz PhD}

\abstract{Treść streszczenia po polsku}
\abstractEN{Treść streszczenia po angielsku}

\begin{document}

% strona tytułowa
\maketitle

\blankpage

% spis treści
\tableofcontents

\clearpage
\blankpage
\clearpage

\section{Wprowadzenie}

\subsection{Przegląd literatury}

\section{Teoria grafów}

\subsection{Rodzaje grafów}

\subsection{Rozpoznawanie grafów}

\section{Wykorzystywane technologie}

\section{Testy}

\subsection{Wyniki}

\subsection{Wnioski}

\section{Podsumowanie i wnioski końcowe}

\clearpage

\section*{Załączniki}

\addcontentsline{toc}{section}{Załączniki}

\clearpage

\addcontentsline{toc}{section}{Literatura}

\begin{thebibliography}{4}
\bibitem{Wilson2000} Wilson R.J.: Wprowadzenie do teorii grafów. PWN, Warszawa 2000.
\bibitem{str} http://student.krk.pl/026-Ciosek-Grybow/rodzaje.html. Dostęp 26.03.2024.
\end{thebibliography}

\clearpage

\makesummary

\end{document} 
