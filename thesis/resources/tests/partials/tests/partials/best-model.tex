W wyniku testów wyłoniony został model o najlepszej dokładności realnej, tj.
dokładności klasyfikacji zewnętrznych grafów testowych.
Model ten okazał się jednym z najbardziej podstawowych z przygotowanych,
bo jest to podstawowy model uczony na grafach sześciowierzchołkowych.
W ramach eksperymentu, zastosowano dla niego modyfikacje,
które pomogły zwiększyć realną dokładność modelu.
Modyfikacje te zostały wprowadzone i przetestowane w podrozdziale z testami modeli
z walidacją krzyżową.
Jedyną zmianą z zaproponowanych, która przynosi realne korzyści,
jest zwiększenie liczby filtrów w warstwach sieci neuronowej, kolejno 32, 64 oraz 128 dla Conv2D,
oraz zastosowanie zwiększonego parametru dropout - 0,5.

\textbf{Zmodyfikowany model podstawowy uczony na grafach sześciowierzchołkowych}

Model osiąga wysokie i stabilne wyniki, lecz po około 55 epoce procesu nauczania,
dokładność gwałtownie spada, aż do wartości około 23\%.

Ta sama sytuacja dotyczy straty, która to gwałtownie wzrasta około 60 epoki,
z około 0,25, do aż 1,75.

\begin{figure}[ht]
	\centering
	\includegraphics[height=5.5cm]{resources/tests/images/v4/base6_1_img.png}
	\caption{Wyniki testów dla zmodyfikowanego modelu podstawowego, liczba wierzchołków n = 6}
	\label{Fig:tests-base-5a}
\end{figure}
\FloatBarrier

Powyższa sytuacja wskazuje na poważny problem z wydajnością modelu,
np. przetrenowanie lub problem techniczny w procesie treningu
- być może był to zbyt niski współczynnik procesu uczenia,
który spowodował, że model „przyspieszył” w ostatnich epokach.

% \begin{figure}[ht]
% 	\centering
% 	\includegraphics[width=14cm]{resources/tests/images/v4/base6_1_txt.png}
% 	\caption{Klasyfikacja obrazów zewnętrznych dla zmodyfikowanego modelu podstawowego, liczba wierzchołków n = 6}
% 	\label{Fig:tests-base-5b}
% \end{figure}
% \FloatBarrier

% \begin{figure}[ht]
% 	\centering
% 	\includegraphics[width=14cm]{resources/tests/images/v4/base6_1_bar.png}
% 	\caption{Wizualizacja klasyfikacji obrazów zewnętrznych dla zmodyfikowanego modelu podstawowego, liczba wierzchołków n = 6}
% 	\label{Fig:tests-base-5c}
% \end{figure}
% \FloatBarrier

\textbf{Zmodyfikowany poprawiony model podstawowy uczony na grafach sześciowierzchołkowych}

Wyniki uzyskane z procesu nauki poprzeniego modelu sugerują zmniejszenie liczby epok do około 55.
Taka modyfikacja została zastosowana, a test przeprowadzono ponownie.

Po skróceniu procesu uczenia, dokładność modelu nie spada gwałtownie w końcowych epokach.
Dokładność szybko rośnie na początku treningu i osiąga wysokie wartości już po kilku epokach.
Dokonuje się również stabilizacja w okolicach 98\%-100\%, co wskazuje na dobre dopasowanie modelu do danych.
Fluktuacje walidacji są niewielkie i nie wskazujące na problemy z modelem.

Strata, po modyfikacji, już nie wzrasta gwłatownie pod koniec procesu uczenia.
Oba typy strat maleją po kilku pierwszych epokach i pozostają na niskim poziomie (0,1 - 0,2).
Model wygląda więc na dobrze dopasowany do danych treningowych, jak i walidacyjnych.

\begin{figure}[ht]
	\centering
	\includegraphics[height=5.5cm]{resources/tests/images/v4/base6_1_1_img.png}
	\caption{Wyniki testów dla poprawionego zmodyfikowanego modelu podstawowego, liczba wierzchołków n = 6}
	\label{Fig:tests-base-6a}
\end{figure}
\FloatBarrier

Wygląda na to, że model efektywnie klasyfikuje grafy ze zbioru walidacyjnego.
Zwracając uwagę na powyższe wskaźniki, można spodziewać się dobrej jakości przewidywań.
Model działa bardzo dobrze, z minimalnymi fluktuacjami, które mogą być naturalne przy tego typu zadaniach.

\begin{figure}[ht]
	\centering
	\includegraphics[width=14cm]{resources/tests/images/v4/base6_1_1_txt.png}
	\caption{Klasyfikacja obrazów zewnętrznych dla poprawionego zmodyfikowanego modelu podstawowego, liczba wierzchołków n = 6}
	\label{Fig:tests-base-6b}
\end{figure}
\FloatBarrier

\begin{figure}[ht]
	\centering
	\includegraphics[width=14cm]{resources/tests/images/v4/base6_1_1_bar.png}
	\caption{Wizualizacja klasyfikacji obrazów zewnętrznych dla poprawionego zmodyfikowanego modelu podstawowego, liczba wierzchołków n = 6}
	\label{Fig:tests-base-6c}
\end{figure}
\FloatBarrier

Zmodyfikowany model poprawnie sklasyfikował prawie 42\% grafów zewnętrznych.
Jest to wynik gorszy o 3 punkty procentowe od wariantu tego samego modelu
z mniejszą liczbą filtrów w warstwach Conv2D oraz niższym wskaźnikiem dropout.
Wynik jednak jest wciąż zadowlający, bo czterdziestodwuprocentowa pewność przy klasyfikacji pięciu klas,
jest wynikiem lepszym niż przypisywanie klas losowo.

Ponownie, to prostszy model okazał się tym, który dokonuje lepszej generalizacji na nowe dane
i lepiej zapamiętuje ogólne wzorce.
Sugeruje to, że zbyt duże skomplikowanie modelu do zadania o niewielkiej złożoności,
przynosi korzyści odwrotne od zamierzonych.