% -- TO DO -- %
W testach zastosowano naukę na maksymalnie 75 epokach.
Analizując krzywe dokładności i straty modeli, można dojść do wniosku,
że dłuższe uczenie nie przyniosłoby pozytywnych skutków lub znikome pozytywne.
Przeprowadzono również testy na mniejszych liczbach epok, lecz wartości takie jak 10, czy 20,
są za małe, by poprawnie nauczyć większość modeli.
Modeli osiągały dość wysokie dokładności już po kilku pierwszych epokach,
ale zdecydowana większość z nich, z biegiem kolejnych epok, nabierała jeszcze lepszej dokładności.

W przypadku jednego modelu, konieczne było zmniejszenie liczby epok do 55,
ze względu na powstające problemy z procesem nauczania, po owej epoce.

Uczenie modeli z wykorzystaniem stałej liczby wierzchołków grafów okazało się bardziej skuteczne
niż nauka na rysunkach grafów o zmiennej liczbie wierzchołków.

Skomplikowaność modelu i zastosowane techniki optymalizacyjne rzadko przekładały się na zwiększoną realną dokładność modelu.

W przypadku uczenia najbardziej zmodyfikowanych modeli, to grafy pełne najczęściej dominowały cały zbiór danych,
przez co modele te testach na danych zewnętrznych,
klasyfikowały większość testowych grafów rysowanych odręcznie jako właśnie grafy pełne.