W przypadku uczenia modeli z wykorzystaniem grafów pełnych, najczęściej dominowały one cały zbiór danych,
przez co modele w kolejnych testach klasyfikowały większość testowych grafów rysowanych odręcznie jako właśnie grafy pełne.

Testy z wykorzystaniem stałej liczby wierzchołków grafów okazały się mniej owocne niż testy z rysunkami grafów
o zmiennej liczbie wierzchołków.

Wystąpiła tendencja do niepoprawnego określania innych grafów, grafami dwudzielnymi, jeśli takie znajdowały się
w zbiorze danych treningowych.