W tym rozdziale zostaną przedstawione podstawowe definicje z~dziedziny teorii grafów.
Definicje zostały zaczerpnięte z~literatury, z~pozycji \cite{Wloch2008}, \cite{Wilson2012} oraz \cite{Wojciechwoski2013}.

\noindent
\textbf{Definicja \definitionIndex.}
\incrementdefinitionIndex
Grafem nieskierowanym, skończonym $G$ nazywamy parę $(V,E)$, gdzie $V = V(G)$ jest zbiorem skończonym, niepustym,
natomiast $E = E(G)$ jest rodziną mogących się powtarzać dwuelementowych podzbiorów niekoniecznie różnych elementów ze zbioru $V$.
Zbiór $V(G)$ nazywamy zbiorem wierzchołków, a~elementy tego zbioru nazywamy wierzchołkami i~oznaczamy symbolami:
$x$, $y$, $x_i$, $y_i$, $1$, $2$, ... Zbiór $E(G)$ nazywamy zbiorem krawędzi grafu $G$.
Mówimy, że krawędź $\{v, w\}$ łączy wierzchołki $v$~i~$w$, i~na ogół oznaczamy ją krócej symbolem $vw$.
W~wielu zagadnieniach nazwy wierzchołków są nieistotne, więc je pomijamy i~mówimy wtedy, że graf jest nieoznakowany.

\noindent
\textbf{Definicja \definitionIndex.}
\incrementdefinitionIndex
Jeżeli w~grafie $G$ istnieją co~najmniej dwie krawędzie $xy$, to~krawędź tę~nazywamy krawędzią wielokrotną.

\noindent
\textbf{Definicja \definitionIndex.}
\incrementdefinitionIndex
Krawędź $xx$ w~grafie $G$ nazywamy pętlą.

\noindent
\textbf{Definicja \definitionIndex.}
\incrementdefinitionIndex
Drogę $P$ z~wierzchołka $x_1$ do~wierzchołka $x_m$ w~grafie $G$ nazywamy skończony ciąg wierzchołków
$x_1, x_2, ..., x_m, m \geqslant 2$ i~krawędzi ${x_i, x_{i + 1}}, i~= 1, ..., m$.

\noindent
\textbf{Definicja \definitionIndex.}
\incrementdefinitionIndex
Odległością pomiędzy wierzchołkami $x$ i~$y$ w~grafie $G$ nazywamy długść najkrótszej drogi łączącej $x$~i~$y$
i~oznaczamy symbolem $d_G(x,y)$.

\noindent
\textbf{Definicja \definitionIndex.}
\incrementdefinitionIndex
Grafem spójnym nazywamy graf $G$, w~którym każde dwa wierzchołki są połaczone drogą dowolnej długości.
Graf, który nie jest spójny, nazywamy grafem niespójnym.

\noindent
\textbf{Definicja \definitionIndex.}
\incrementdefinitionIndex
Dwa wierzchołki $x$, $y$ w~grafie $G$ są sąsiednie, jeżeli $xy \in E(G)$.
Mówimy wtedy, żę wierzchołki $x$ i~$y$ są incydentne z~tą krawędzią.

\noindent
\textbf{Definicja \definitionIndex.}
\incrementdefinitionIndex
Stopień wierzchołka $v$ oznaczany symbolem $deg(v)$ jest liczbą krawędzi incydentnych z~$v$.

\noindent
\textbf{Definicja \definitionIndex.}
\incrementdefinitionIndex
Wierzchołek stopnia 0 nazywamy wierzchołkiem izolowanym, a~wierzchołek stopnia 1, liściem.

\noindent
\textbf{Definicja \definitionIndex.}
\incrementdefinitionIndex
Graf $G$ taki, że $E(G) = \emptyset$, nazywamy grafem bezkrawędziowym. Jeżeli $|V(G)| = n$, to~graf bezkrawędziowy oznaczony symbolem $N_n$.
Każdy wierzchołek grafu bezkrawędziowego jest wierzchołkiem izolowanym.

\noindent
\textbf{Definicja \definitionIndex.}
\incrementdefinitionIndex
Graf prosty $G$ taki, że każde dwa wierzchołki są sąsiednie, nazywamy grafem pełnym.
Jeżeli $|V(G)| = n$, to~graf pełny oznaczamy $K_n$.

\noindent
\textbf{Definicja \definitionIndex.}
\incrementdefinitionIndex
Graf skierowany - graf niezawierający krawędzi niezorientowanych.

\noindent
\textbf{Definicja \definitionIndex.}
\incrementdefinitionIndex
Drogę, w której $x_m = x_1, m \geqslant 3$, nazywamy cyklem $C$.

\noindent
\textbf{Definicja \definitionIndex.}
\incrementdefinitionIndex
Przodek to~każdy inny wierzchołek, z~którego istnieje ścieżka prowadząca do~danego wierzchołka.
Innymi słowy, przodek to~wierzchołek, który znajduje się wyżej w hierarchii,
patrząc od strony korzenia w kierunku danego wierzchołka w grafie skierowanym lub~drzewie.

\noindent
\textbf{Definicja \definitionIndex.}
\incrementdefinitionIndex
Korzeń w~drzewie jest jedynym wierzchołkiem,
który nie ma przodka, wszystkie wierzchołki sąsiednie z~korzeniem są jego potomkami.

\noindent
\textbf{Definicja \definitionIndex.}
\incrementdefinitionIndex
Drzewem nazywany spójny graf bez cykli.

\noindent
\textbf{Definicja \definitionIndex.}
\incrementdefinitionIndex
Poddrzewem drzewa z~korzeniem $T$ nazywamy drzewo, którego korzeniem jest wierzchołek $x_k, x_k \neq x_1$,
które zawiera wszystkich potomków wierzchołka $x_k$ z~drzewa $T$.

\noindent
\textbf{Definicja \definitionIndex.}
\incrementdefinitionIndex
Drzewem binarnym nazywamy drzewo składające się z~wyróżnionego wierzchołka nazywanego korzeniem oraz dwóch poddrzew
- lewgo $T_l$ oraz prawego $T_p$.