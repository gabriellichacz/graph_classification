\addcontentsline{toc}{section}{Literatura}

\begin{thebibliography}{4}
    \bibitem{Arikawa2019} Arikawa K.: Graph Theory Teaches Us Something About Grammaticality. The Prague Bulletin of Mathematical Linguistics No. 112, 2019, pp. 55-82
    \bibitem{Balaban1985} Balaban A.T.: Applications of Graph Theory in Chemistry. Department of Organic Chemistry, Polytefhnic Institute. 76206 Bucharest, Roumania, 1985
    \bibitem{Brusatte2021} Brusatte, S., Carr, T. The phylogeny and evolutionary history of tyrannosauroid dinosaurs. Sci Rep 6, 20252 (2016). https://doi.org/10.1038/srep20252
    \bibitem{Chung2021} Chung M.K.: Graph Theory in Brain Networks, University of Wisconsin-Madison, 2021 
    \bibitem{Erciyes2023} Erciyes K.: Graph-Theoretical Analysis of Biological Networks: A Survey. Computation 2023, 11, 188, DOI: https://doi.org/10.3390/computation11100188
    \bibitem{Fenner2020} Fenner M.E.: Uczenie maszynowe w Pythonie dla każdego. Helion SA, Gliwice 2020.
    \bibitem{Geron2020} Géron A.: Uczenie maszynowe z użyciem Scikit-Learn i TensorFlow. Helion SA, Gliwice 2020.
    \bibitem{Goodfellow2016} Goodfellow I., Bengio Y., Courville A.: Deep Learning, MIT Press 2016
    \bibitem{Harary1953} Harary F., Norman R.Z.: Graph Theory as a Mathematical Model in Social Science, Research Center for Group Dynamics, University of Michigan, 1953
    \bibitem{Rodak2021} Rodak K., Kokot I., Kratz E.W.: Caffeine as a Factor Influencing the Functioning of the Human Body-Friend or Foe? Nutrients. 2021 Sep 2;13(9):3088
    \bibitem{Seenappa} Seenappa M.G.: Graph Classification using Machine Learning Algorithms. Master's Projects. 725, San Jose State University 2019, DOI: https://doi.org/10.31979/etd.b9pm-wpng
    \bibitem{Umami2024} Umami M.H., Prihandini R.M., Agatha A.B.: Application of Graph Theory to Social Network Analysis, Department of Mathematics Educations, University of Jember, Jember, Indonesia, 2024
    \bibitem{vanDerMaaten} L.J.P. van der Maaten, Hinton G.E.: Visualizing High-Dimensional Data Using t-SNE. Journal of Machine Learning Research 9(Nov):2579-2605, 2008
    \bibitem{Wilson2012} Wilson R.J.: Wprowadzenie do teorii grafów. PWN, Warszawa 2012.
    \bibitem{Wloch2008} Włoch A., Włoch I.: Matematyka dyskretna. Podstawowe metody i algorytmy teorii grafów. Oficyna Wydawnicza Politechniki Rzeszowskiej, Rzeszów 2008.
    \bibitem{Wojciechwoski2013} Wojciechowski J., Pieńkosz K.: Grafy i sieci. PWN, Warszawa 2013.
    \bibitem{str3} https://www.kaggle.com/datasets/robikscube/flight-delay-dataset-20182022. Dostęp 06.01.2024.
    \bibitem{str3} https://cran.r-project.org/web/packages/igraph/index.html. Dostęp 10.03.2024.
    \bibitem{str5} https://developers.google.com/machine-learning. Dostęp 20.07.2024.
    \bibitem{str4} https://www.ibm.com/topics. Dostęp 20.07.2024.
    \bibitem{strPython} https://www.python.org/. Dostęp 07.08.2024.
    \bibitem{strR} https://www.r-project.org/. Dostęp 07.08.2024.
    \bibitem{str1} http://student.krk.pl/026-Ciosek-Grybow/rodzaje.html. Dostęp 26.03.2024.
    \bibitem{str6} https://www.tensorflow.org/api\_docs. Dostęp 21.07.2024.
\end{thebibliography}