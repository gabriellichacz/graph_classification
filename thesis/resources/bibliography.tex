\addcontentsline{toc}{chapter}{Literatura}

\begin{thebibliography}{4}
    \bibitem{Arikawa2019} Arikawa K.: \textit{Graph Theory Teaches Us Something About Grammaticality}. The Prague Bulletin of Mathematical Linguistics 112, 55-82, 2019.
    \bibitem{Balaban1985} Balaban A.T.: \textit{Applications of Graph Theory in Chemistry}. Department of Organic Chemistry Polytechnic Institute, Bucharest, 1985.
    \bibitem{Brusatte2021} Brusatte, S., Carr, T.: \textit{The phylogeny and evolutionary history of tyrannosauroid dinosaurs}. Scientific Reports 6, 20252, 2016.
    \bibitem{Chung2021} Chung M.K.: \textit{Graph Theory in Brain Networks}, University of Wisconsin-Madison, Madison, 2021.
    \bibitem{Erciyes2023} Erciyes K.: \textit{Graph-Theoretical Analysis of Biological Networks: A Survey}. Computation 11, 188, 2023.
    \bibitem{Fenner2020} Fenner M.E.: \textit{Uczenie maszynowe w Pythonie dla każdego}. Helion SA, Gliwice, 2020.
    \bibitem{Geron2020} Géron A.: \textit{Uczenie maszynowe z użyciem Scikit-Learn i TensorFlow}. Helion SA, Gliwice, 2020.
    \bibitem{Goodfellow2016} Goodfellow I., Bengio Y., Courville A.: \textit{Deep Learning}, MIT Press, Cambridge, 2016.
    \bibitem{Harary1953} Harary F., Norman R.Z.: \textit{Graph Theory as a Mathematical Model in Social Science}, Research Center for Group Dynamics University of Michigan, Michigan, 1953.
    \bibitem{Rodak2021} Rodak K., Kokot I., Kratz E.W.: \textit{Caffeine as a Factor Influencing the Functioning of the Human Body-Friend or Foe?} Nutrients 13, 3088, 2021.
    \bibitem{Seenappa} Seenappa M.G.: \textit{Graph Classification using Machine Learning Algorithms}. San Jose State University, San Jose, 2019.
    \bibitem{Umami2024} Umami M.H., Prihandini R.M., Agatha A.B.: \textit{Application of Graph Theory to Social Network Analysis}, Department of Mathematics Educations University of Jember, Jember, 2024.
    \bibitem{vanDerMaaten} L.J.P. van der Maaten, Hinton G.E.: \textit{Visualizing High-Dimensional Data Using t-SNE}. Journal of Machine Learning Research 9, 2579-2605, 2008.
    \bibitem{Wilson2012} Wilson R.J.: \textit{Wprowadzenie do teorii grafów}. PWN, Warszawa, 2012.
    \bibitem{Wloch2008} Włoch A., Włoch I.: \textit{Matematyka dyskretna. Podstawowe metody i algorytmy teorii grafów}. Oficyna Wydawnicza Politechniki Rzeszowskiej, Rzeszów, 2008.
    \bibitem{Wojciechwoski2013} Wojciechowski J., Pieńkosz K.: \textit{Grafy i sieci}. PWN, Warszawa, 2013.
    % - TO DO -- % Sprawdzić wywołania
    \bibitem{str3} https://www.kaggle.com/datasets/robikscube/flight-delay-dataset-20182022. Dostęp 06.01.2024.
    \bibitem{str3} https://cran.r-project.org/web/packages/igraph/index.html. Dostęp 10.03.2024.
    \bibitem{str5} https://developers.google.com/machine-learning. Dostęp 20.07.2024.
    \bibitem{str4} https://www.ibm.com/topics. Dostęp 20.07.2024.
    \bibitem{strPython} https://www.python.org/. Dostęp 07.08.2024.
    \bibitem{strR} https://www.r-project.org/. Dostęp 07.08.2024.
    \bibitem{str1} http://student.krk.pl/026-Ciosek-Grybow/rodzaje.html. Dostęp 26.03.2024.
    \bibitem{str6} https://www.tensorflow.org/api\_docs. Dostęp 21.07.2024.
    \bibitem{str2} http://wms.mat.agh.edu.pl/\texttildelow md/ang-pol.pdf. Dostęp 29.03.2024.
\end{thebibliography}