Według \cite{Geron2020}, uczenie maszynowe można sklasyfikować na podstawie kilku kryteriów.
Jest to nadzór człowieka w~procesie trenowania, możliwość modelu do uczenia się w~czasie rzeczywistym
oraz sam sposób pracy (nauka z~przykładów lub modelu). Kryteria te nie wykluczają się wzajemnie - można je dowolnie łączyć.
Za przykład może posłużyć filtr antyspamowy, który ciągle się uczy,
wykorzystując model sieci neuronowej i~analizując wiadomości email.
Taki system można określić przyrostowym, opartym na modelu i~nadzorowanym.

Dodatkowe kryteria oceny rodzaju uczenia maszynowego:
\begin{itemize}[label=-,labelsep=0.4cm,leftmargin=0.6cm]
    \item Uczenie nadzorowane (ang. supervised learning) to podejście, w~którym model jest szkolony na danych,
        które są już odpowiednio oznaczone (np. rekordy mają przypisane odpowiednie klasy).
        Celem jest odkrycie funkcji, która przekształca dane wejściowe w~oczekiwane wyjścia.
        Znajduje zastosowanie w~klasyfikacji i~regresji.
        Przykład: Klasyfikacja wiadomości e-mail jako spam lub nie-spam.
    \item Uczenie nienadzorowane (ang. unsupervised learning)
        - w~tym przypadku model bada nieoznaczone dane, aby odkryć pewne wzorce lub struktury.
        Najczęściej stosowane w~klasteryzacji, czy redukcji wymiarowości. Przykłady:
        \begin{itemize}[label=*,labelsep=0.4cm,leftmargin=0.8cm]
            \item Klasteryzacja klientów w~celu segmentacji rynku, gdzie klienci są grupowani na podstawie ich zachowań zakupowych. 
            \item Redukcja wymiarowości w~celu wizualizacji danych wysokowymiarowych, np. za pomocą algorytmu t-SNE.
        \end{itemize}
    \item Uczenie przez wzmacnianie (ang. reinforcement learning) to przypadek,
        gdzie model uczy się poprzez interakcję ze swoim otoczeniem,
        podejmując decyzje, które maksymalizują pewną nagrodę. Przykłady:
        \begin{itemize}[label=*,labelsep=0.4cm,leftmargin=0.8cm]
            \item Algorytmy sterujące robotami, które uczą się poruszać w~nieznanym terenie. 
            \item Programy grające w~gry, takie jak AlphaGo (chińska gra Go), które uczą się strategii gry poprzez rozgrywanie wielu partii.
        \end{itemize}
\end{itemize}