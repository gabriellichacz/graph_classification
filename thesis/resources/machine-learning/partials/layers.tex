Głębokie sieci neuronowe, opierają się na~strukturze warstw,
które przekształcają i~analizują dane wejściowe, aby uzyskać z~nich pożądane cechy.
Każda z~tych warstw pełni pewną, specyficzną funkcję, począwszy od przeskalowania danych,
przez konwolucje (przekształcenie macierzowe danych),
aż po~bardziej zaawansowane operacje, umożliwiając modelowi sprawniejsze uczenie się bardziej złożonych wzorców.
\begin{itemize}[label=-,labelsep=0.4cm,leftmargin=0.6cm]   
	\item Warstwa Flatten (spłaszczona) ma za zadanie przekształcić każdy obraz wejściowy w~tablicę jednowymiarową.
		Nie zawiera żadnych parametrów, a~jej jedynym celem jest proste, wstępne przetworzenie danych \cite{Geron2020}.
	\item Warstwa Dense (w pełni połączona) zarządza samodzielnie swoją macierzą wag,
		zawierającą wszystkie wagi połączeń między neuronami, a~wejściami do~nich oraz wekorem obciążeń.
		Zawiera najczęściej bardzo dużo parametrów, dzięki czemu model uzyskuje swobodę w~dopasowaniu do~danych treningowych.
		Jednocześnie, grozi mu również przez to~ryzyko przetrenowania,
		zwłaszcza w~przypadku korzystania z~mniejszych zestawów danych \cite{Geron2020}.
    \item Warstwa Rescaling mnoży każde wejście przez ustalony współczynnik skalujący.
		Zastosowanie tej techniki jest przydatne, gdy różne cechy danych wejściowych mają różne zakresy wartości.
		Poprzez jednolite skalowanie, model może efektywniej uczyć się wzorców, a~proces optymalizacji staje się stabilniejszy \cite{Fenner2020}.
	\item Warstwa Conv2D tworzy jądro splotu, które jest nakładane na~dane wejściowe w~jednym wymiarze przestrzennym (lub czasowym), 
		aby wygenerować tensor danych wyjściowych.
		Dodatkowo, jeśli stosowana jest funkcja aktywacji, jest ona stosowana również do~danych wyjściowych \cite{Geron2020}.
	\item Warstwa MaxPooling2D dokonuje redukcji wymiarów danych wejściowych wzdłuż ich wymiarów przestrzennych (wysokości i~szerokości),
		wybierając maksymalną wartość z~każdego okna o~rozmiarze określonym przez wybrany współczynnik $pool\_size$,
		dla każdego kanału danych wejściowych.
		Okno to~jest przesuwane o~określoną liczbę kroków wzdłuż obu wymiarów \cite{Geron2020}.
	\item Warstwa Dropout losowo zeruje jednostki wejściowe z~prawdopodobieństwem określonym
		przez wybrany współczynnik dropout na~każdym etapie treningu, co~pomaga unikać przeuczenia modelu.
		Jednostki, które nie zostały wyzerowane, są skalowane w~górę przez mnożenie przez $\frac{1}{1 - współczynnikDropout}$,
		aby suma wartości wejściowych pozostała niezmieniona \cite{Geron2020}.
\end{itemize}